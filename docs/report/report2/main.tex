\documentclass{article}
\usepackage[utf8]{inputenc}
\usepackage{float}
\usepackage{xcolor}
\usepackage{graphicx}

\usepackage{hyperref}
\hypersetup{
    colorlinks=true,
    linkcolor=blue,
    filecolor=magenta,      
    urlcolor=blue,
}

\title{Scientific Computing - Molecular dynamics \\ Group F}
\newcommand{\subtitle}{Problem sheet 2}
\author{
    Jimin Kim \\
    Christian Nix \\
    Noah Schlenker
}
\date{\today}

\begin{document}

\maketitle

\begin{center}
    \LARGE \subtitle{}
\end{center}

\section{Pull request}
The pull request can be found \href{https://github.com/noahpy/MolSim-SS24/pull/10}{here}.

\section{Unit Tests}

\begin{itemize}
    \item We already successfully implemented \verb|gtest| in the previous assignment, providing a robust foundation for subsequent testing enhancements.
    \item Integration of \verb|gtest| through CMake is managed within its dedicated CMake module, reinforcing a decentralized and modular system architecture.
    \item The system verifies the presence of \verb|gtest| before attempting to fetch the library via CMake.
    \item Building upon the tests included in the previous assignment, the additions for this iteration solely encompass tests for the force calculation via the Lennard-Jones potential and the particle generator functionality.
    \item The tests for the Lennard-Jones force calculation perform direct verification of computational accuracy (see \texttt{tests/physics/testForceLJ.cpp}).
    \item Tests for the particle generator validate the structural integrity of generated cuboids and the properties of contained particles \newline(see \texttt{tests/simulation/testParticleGenerator.cpp}).
\end{itemize}

\section{Realization of Continuous Integration}

\begin{itemize}
    \item We implemented CI through Docker, ensuring a consistent and isolated environment for all integration and testing processes.
    \item We utilized `nektos/act`, a tool that simulates GitHub Actions locally, allowing us to validate workflows before pushing to the repository.
    \item The CI pipeline was expanded to include dynamic analysis with the GCC `-fsanitize=address` flag and integrated unit tests within the CI pipeline to verify that all tests pass before any code merges into the master or 'assignment{number}' branches, enhancing code reliability.
    \item Branch protection rules were successfully configured, ensuring no direct pushes can occur and all merges require successful CI checks and pull-request reviews.
\end{itemize}

\section{Logging Configuration via spdlog}

\begin{itemize}
    \item In alignment with our methodology for integrating \verb|gtest|, \verb|spdlog| was incorporated using a CMake module, including a pre-fetch check to ensure its availability before fetching it via CMake.
    \item We opted to utilize the functions provided by \verb|spdlog|, favoring their type safety and straightforward integration, which aids in more explicit and clear code management and debugging. This approach ensures that each logging call is checked at runtime, maintaining flexibility while slightly increasing overhead.
    \item The logging level can be dynamically adjusted via the command-line option \texttt{-l}, facilitating the ease of toggling between different logging levels according to the needs of runtime diagnostics.
    \item Although \verb|spdlog| macros offer potential performance enhancements by excluding logging code at compile-time for disabled log levels, we determined that this advantage was not critical for our current project scope. The slight performance gain did not outweigh the benefits of using functions for our purposes. However, we may reconsider this decision in future assignments if a significant performance difference is observed.
\end{itemize}




\end{document}
