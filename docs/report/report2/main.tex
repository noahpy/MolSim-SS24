\documentclass{article}
\usepackage[utf8]{inputenc}
\usepackage{float}
\usepackage{xcolor}
\usepackage{graphicx}

\usepackage{hyperref}
\hypersetup{
    colorlinks=true,
    linkcolor=blue,
    filecolor=magenta,      
    urlcolor=blue,
}

\title{Scientific Computing - Molecular dynamics \\ Group F}
\newcommand{\subtitle}{Problem sheet 2}
\author{
    Jimin Kim \\
    Christian Nix \\
    Noah Schlenker
}
\date{\today}

\begin{document}

\maketitle

\begin{center}
    \LARGE \subtitle{}
\end{center}

\section{Pull request}
\label{sec:pr}
The pull request can be found \href{https://github.com/noahpy/MolSim-SS24/pull/10}{here}.

\section{Unit Tests}
\label{sec:ut}

\begin{itemize}
    \item We already successfully implemented \verb|gtest| in the previous assignment, providing a robust foundation for subsequent testing enhancements.
    \item Integration of \verb|gtest| through CMake is managed within its dedicated CMake module, reinforcing a decentralized and modular system architecture.
    \item The system verifies the presence of \verb|gtest| before attempting to fetch the library via CMake.
    \item Building upon the tests included in the previous assignment, the additions for this iteration solely encompass tests for the force calculation via the Lennard-Jones potential and the particle generator functionality.
    \item The tests for the Lennard-Jones force calculation perform direct verification of computational accuracy (see \texttt{tests/physics/testForceLJ.cpp}).
    \item Tests for the particle generator validate the structural integrity of generated cuboids and the properties of contained particles \newline(see \texttt{tests/simulation/testParticleGenerator.cpp}).
\end{itemize}

\section{Realization of Continuous Integration}
\label{sec:ci}

\begin{itemize}
    \item We implemented CI through Docker, ensuring a consistent and isolated environment for all integration and testing processes.
    \item We utilized `nektos/act`, a tool that simulates GitHub Actions locally, allowing us to validate workflows before pushing to the repository.
    \item The CI pipeline was expanded to include dynamic analysis with the GCC `-fsanitize=address` flag and integrated unit tests within the CI pipeline to verify that all tests pass before any code merges into the master or 'assignment{number}' branches, enhancing code reliability.
    \item Branch protection rules were successfully configured, ensuring no direct pushes can occur and all merges require successful CI checks and pull-request reviews.
\end{itemize}

\section{Logging Configuration via spdlog}
\label{sec:spd}

\begin{itemize}
    \item In alignment with our methodology for integrating \verb|gtest|, \verb|spdlog| was incorporated using a CMake module, including a pre-fetch check to ensure its availability before fetching it via CMake.
    \item We opted to utilize the functions provided by \verb|spdlog|, favoring their type safety and straightforward integration, which aids in more explicit and clear code management and debugging. This approach ensures that each logging call is checked at runtime, maintaining flexibility while slightly increasing overhead.
    \item The logging level can be dynamically adjusted via the command-line option \texttt{-l}, facilitating the ease of toggling between different logging levels according to the needs of runtime diagnostics.
    \item Although \verb|spdlog| macros offer potential performance enhancements by excluding logging code at compile-time for disabled log levels, we determined that this advantage was not critical for our current project scope. The slight performance gain did not outweigh the benefits of using functions for our purposes. However, we may reconsider this decision in future assignments if a significant performance difference is observed.
\end{itemize}

\section{Adjustment of program frame}
\label{sec:adj}

\subsection{I/O Expansion}
\label{subsec:file}

\begin{itemize}
    \item We have enhanced our I/O functionalities to support various input file formats.
    \item Introduction of specialized file-reader functions enabled us to handle different data types, including cluster and ASCII art data. These functions extend the capabilities of the base reader function (see \texttt{src/io/fileReader}).
    \item We implemented a factory function that returns a unique pointer to either a specific file-reader or the default file-reader, depending on the input file format.
    \item Additionally, a factory function was included to facilitate switching between \texttt{xyz} and \texttt{vtk} file writers, enhancing our output flexibility.
\end{itemize}

\subsection{Simulation Expansion}
\label{subsec:sim}

\begin{itemize}
    \item The simulation for using the Lennard-Jones-potential was included.
    \item We included a factory function for the simulations that allows for interchangeable scenarios.
\end{itemize}

\subsection{Additional Arguments for the Program}
\label{subsec:arg}

\begin{itemize}
    \item We have included several additional command-line options in response to recent changes in our program.
    \item The parameters \texttt{epsilon} and \texttt{sigma} can now be specified via command-line if the default values are not suitable for the user's needs.
    \item Additional command-line arguments have been introduced to allow users to specify the input file format, supporting both clusters and ASCII art (options \texttt{-c} or \texttt{-a}).
    \item Users now have the flexibility to set the simulation parameters with \texttt{-s} and the file-writer type with \texttt{-w} after compiling, enhancing runtime configurability.
\end{itemize}


\section{Inclusion of Man page}
\label{sec:man}

\begin{itemize}
    \item We included a man page, which can be accessed by executing the command \texttt{man ./molsim.1} from the root directory of the project.
    \item It includes more specified details for the input formats and the functionalities of our program.
\end{itemize}


\end{document}
