\documentclass{article}
\usepackage[utf8]{inputenc}
\usepackage{float}
\usepackage{xcolor}
\usepackage{graphicx}
\usepackage{amsmath}
\usepackage{amssymb}
\usepackage{placeins}
\usepackage{booktabs}
\usepackage{caption}

\usepackage{hyperref}
\hypersetup{
    colorlinks=true,
    linkcolor=blue,
    filecolor=magenta,      
    urlcolor=blue,
}

\title{Scientific Computing - Molecular dynamics \\ Group F}
\newcommand{\subtitle}{Problem sheet 4}
\author{
    Jimin Kim \\
    Christian Nix \\
    Noah Schlenker
}
\date{\today}

\begin{document}

\maketitle

\begin{center}
    \LARGE \subtitle{}
\end{center}

\section{Pull request}
\label{sec:pr}
The pull request can be found \href{https://github.com/noahpy/MolSim-SS24/pull/42}{here}.

\section{Bug Fixes and Implementation of missing features}
\label{sec:fix}

    \begin{itemize}
        \item We addressed several issues and incorporated the missing functionalities from the previous assignment.
        \item A check for the cutoff radius in particle calculations has been added.
        \item We implemented the removal of particles from calculations and ParaView visualization if they move beyond the outflow boundaries (see previous report for our prior handling of outflow boundaries):
        \begin{itemize}
            \item Particles now possess a boolean attribute, \texttt{active}, indicating whether the particle is out of bounds.
            \item Instead of performing costly operations and reallocations to delete particles, we manage them using simple flags.
            \item The refactored project structure allows us to modify particle container iterators without disrupting the functionalities of previous assignments or necessitating extensive code adjustments.
        \end{itemize}
        \item We corrected errors affecting certain particles at the corners of reflective boundaries.
        \item The simulation refactoring to allow selection between different simulation options without creating a new simulation class will be implemented in the next assignment. This decision was made to prioritize the new tasks and due to time constraints.
    \end{itemize}

All new features have been tested with their according Unit tests (see \texttt{tests}).

\section{Thermostat}
\label{sec:thermo}

    \begin{itemize}
        \item We introduced a new class, \texttt{Thermostat}, in \texttt{src/physics} for temperature calculations to adjust the velocities of all particles in the container.
        \item The temperature update process follows the calculation steps outlined in the assignment sheet.
        \item To optimize temperature calculations, we excluded the divisions by 2 in the calculation of $E_{kin}$, as the energy is multiplied by 2 in the current temperature calculation:
        \begin{itemize}
            \item $T_{current} = \frac{2 \cdot \left(\sum_{i=1}^{\#particles} \frac{m_i \langle v_i, v_i \rangle}{2}\right)}{\#dimensions\ \cdot\ \#particles}$
            \item $T_{current} = \frac{2 \cdot \frac{1}{2}\ \left(\sum_{i=1}^{\#particles} m_i \langle v_i, v_i \rangle\right)}{\#dimensions\ \cdot\ \#particles}$
            \item $T_{current} = \frac{\sum_{i=1}^{\#particles} m_i \langle v_i, v_i \rangle}{\#dimensions\ \cdot\ \#particles}$
        \end{itemize}
        \item This class also provides a function for temperature initialization utilizing the \texttt{maxwellBoltzmannDistributedVelocity} function.
        \item The step counter for the frequency of temperature updates is managed within the simulation layer, as it controls the frequency of update operations in our project structure.
    \end{itemize}
    
    
\section{Rayleigh-Taylor instability Simulation}
\label{sec:rayleigh}

    \begin{itemize}
        \item Periodic Boundary:
        \begin{itemize}
            \item The periodic boundary is realized by duplicating the particle about to leave the boundary into a halo cell.
            \item We map particles from one side to the opposite side by generating a translation map.
        \end{itemize}
        \item XML related parser and reader have been adjusted to accommodate new parameters, such as particle types in clusters.
        \item A new force calculation function, \texttt{force\_mixed\_LJ\_gravity\_lc}, incorporating an additional gravitational force, has been implemented.
        \item We introduced a new simulation class, \texttt{MixedLJSimulation}, for simulating different particle types simultaneously:
        \begin{itemize}
            \item The Lennard-Jones parameters, following to the Lorentz-Berthelot mixing rule, are determined in the constructor of the class.
            \item $\epsilon$ and $\sigma$ values of all type combinations, along with alphas, betas and gammas, are computed and inserted upon simulation initialization.
            \item The simulation checks if particles are outside the domain during initialization.
            \item We included the thermostat, so that the temperature initialization is handled during simulation setup.
        \end{itemize}
        \item The added section of the project UML diagram can be seen here\ \ref{fig:uml}.
        \item Running the simulation revealed a couple observations:
        \begin{itemize}
            \item The falling particles bounce noticeably, when the two particle layers collide.
            \item Initially, the top liquid doesn't mix into the bottom layer evenly, but forms several drops as it sinks.
        \end{itemize}
    \end{itemize}

\begin{figure}[H]
    \includegraphics[width=1.25\textwidth]{res/UML4v2.drawio}
    \caption{UML diagram extension.}
    \label{fig:uml}
\end{figure}

\section{Falling drop - Liquid}
\label{sec:drop}

    \begin{itemize}
        \item Checkpoints were implemented to acquire the equilibrated fluid:
        \begin{itemize}
            \item Checkpointing is realized through saving a simulation state in an XML file.
            \item For that we had to change the XSD-format, to allow saving particle information and specify the starting time in the XML files.
            \item We implemented a \texttt{XmlWriter} for writing the formatted files (see \texttt{src/io/fileWriter}), which can be activated via the command line argument \texttt{-w 2}.
        \end{itemize}
        \item We first applied the thermostat to the simulation, which showed a slow descend of the drop and a calm assimilation into the fluid below due to velocity adjustments.
        \item The drop falls rapidly and splashes violently into liquid without temperature regulation.
        \item Running the simulation with periodic boundaries showed us the following differences:
    \end{itemize}

\end{document}
