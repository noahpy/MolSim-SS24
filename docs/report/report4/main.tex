\documentclass{article}
\usepackage[utf8]{inputenc}
\usepackage{float}
\usepackage{xcolor}
\usepackage{graphicx}
\usepackage{amsmath}
\usepackage{amssymb}
\usepackage{placeins}
\usepackage{booktabs}
\usepackage{caption}

\usepackage{hyperref}
\hypersetup{
    colorlinks=true,
    linkcolor=blue,
    filecolor=magenta,      
    urlcolor=blue,
}

\title{Scientific Computing - Molecular dynamics \\ Group F}
\newcommand{\subtitle}{Problem sheet 4}
\author{
    Jimin Kim \\
    Christian Nix \\
    Noah Schlenker
}
\date{\today}

\begin{document}

\maketitle

\begin{center}
    \LARGE \subtitle{}
\end{center}

\section{Pull request}
\label{sec:pr}
The pull request can be found \href{https://github.com/noahpy/MolSim-SS24/pull/42}{here}.

\section{Fixes for last assignment}
\label{sec:fix}

    \begin{itemize}
        \item We addressed several issues and incorporated the missing functionalities from the previous assignment.
        \item A check for the cutoff radius in particle calculations has been added.
        \item We implemented the removal of particles from calculations and ParaView visualization if they move beyond the outflow boundaries (see previous report for our prior handling of outflow boundaries):
        \begin{itemize}
            \item Particles now possess a boolean attribute, \texttt{active}, indicating whether the particle is out of bounds.
            \item Instead of performing costly operations and reallocations to delete particles, we manage them using simple flags.
            \item The refactored project structure allows us to modify particle container iterators without disrupting the functionalities of previous assignments or necessitating extensive code adjustments.
        \end{itemize}
        \item We corrected errors affecting certain particles at the corners of reflective boundaries.
    \end{itemize}

Unit tests and XML adaptions are provided for the following sections.

\section{Thermostat}
\label{sec:thermo}

    \begin{itemize}
        \item We introduced a new class, \texttt{Thermostat}, in \texttt{src/physics} for temperature calculations to adjust the velocities of all particles in the container.
        \item The temperature update process follows the calculation steps outlined in the assignment sheet.
        \item To optimize temperature calculations, we excluded the divisions by 2 in the calculation of $E_{kin}$, as the energy is multiplied by 2 in the current temperature calculation:
        \begin{itemize}
            \item $T_{current} = \frac{2 \cdot \left(\sum_{i=1}^{\#particles} \frac{m_i \langle v_i, v_i \rangle}{2}\right)}{\#dimensions\ \cdot\ \#particles}$
            \item $T_{current} = \frac{2 \cdot \frac{1}{2}\ \left(\sum_{i=1}^{\#particles} m_i \langle v_i, v_i \rangle\right)}{\#dimensions\ \cdot\ \#particles}$
            \item $T_{current} = \frac{\sum_{i=1}^{\#particles} m_i \langle v_i, v_i \rangle}{\#dimensions\ \cdot\ \#particles}$
        \end{itemize}
        \item Temperature initialization occurs during particle generation and is not handled within this class.
        \item The step counter for the frequency of temperature updates is managed within the simulation layer, as it controls the frequency of update operations in our project structure.
    \end{itemize}
    
    
\section{Rayleigh-Taylor instability Simulation}
\label{sec:rayleigh}

    \begin{itemize}
        \item Periodic Boundary:
        \begin{itemize}
            \item The periodic boundary is realized via duplication of the particle about to leave the particle to a halo cell.
            \item We map particles from one side to the opposite side by generating a translation map.
        \end{itemize}
    \end{itemize}

\end{document}
