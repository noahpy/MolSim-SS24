\documentclass{article}
\usepackage[utf8]{inputenc}
\usepackage{float}
\usepackage{xcolor}
\usepackage{graphicx}
\usepackage{amsmath}
\usepackage{amssymb}
\usepackage{placeins}
\usepackage{booktabs}
\usepackage{caption}
\usepackage{makecell}

\usepackage{hyperref}
\usepackage{textcomp}
\hypersetup{
    colorlinks=true,
    linkcolor=blue,
    filecolor=magenta,      
    urlcolor=blue,
}

\title{Scientific Computing - Molecular dynamics \\ Group F}
\newcommand{\subtitle}{Problem sheet 5}
\author{
    Jimin Kim \\
    Christian Nix \\
    Noah Schlenker
}
\date{\today}

\begin{document}

\maketitle

\begin{center}
    \LARGE \subtitle{}
\end{center}

\section{Pull request}
\label{sec:pr}
The pull request can be found \href{https://github.com/noahpy/MolSim-SS24/pull/60}{here}.

All new features have been tested with their according Unit tests (see \texttt{tests}) and have XML adaptations respectively.

\section{Membrane simulation}
\label{sec:mem}

    \begin{itemize}
        \item To achieve efficient membrane particle iteration, we utilized maps of neighboring molecular particles in the new \texttt{Membrane} class.
        \item We introduced a unique ID attribute for particles, identifying them within a molecule. This allowed us to store only the molecular IDs in the neighbor maps, which is more memory-efficient than storing reference wrappers.
        \item We separated direct and diagonal neighbors into distinct maps for quick differentiation when calculating harmonic potentials.
        \item By storing only the top, top-right, right, and bottom-right neighbors (half of the neighbors) of a molecular particle, we enabled force calculation iteration using Newton's third law, thus reducing the number of calculations.
        \item The iteration of membrane particles can be conceptualized as a directional iteration through a non-cyclic graph with a single root, marked by a boolean flag within the particle (illustrated here\ \ref{fig:mem}).
        \item We added functions for the calculation of truncated intra-molecular and harmonic forces. The Lennard-Jones potential function was reused from the previous assignment.
        \item Our usual particle generation of our program structure proved to be incompatible for the membrane realization, as we needed to make the neighbor maps accessible at the simulation level.
        \item Therefore, the membrane cluster generation is managed by the \texttt{Membrane} class itself, eliminating the need for cluster specifications in the XML input file.
        \item The new simulation class, \texttt{MembraneSimulation}, extends \texttt{MixedLJSimulation} and manages the membrane initialization and generation accordingly.
    \end{itemize}

\begin{figure}[H]
    \centering
    \includegraphics[width=0.3\textwidth]{../../res/membraneNeighbor.drawio}
    \caption{Illustration of neighbor relations in the membrane.}
    \label{fig:mem}
\end{figure}

\section{Parallelization}
\label{sec:para}

    \begin{itemize}
        \item The parallelization of our program primarily focused on iterating particles for calculations.
        \item A mutex was added to particles to prevent race conditions.
        \item We encountered issues with the parallelization of force calculations due to the handling of cell iteration, requiring significant adjustments to our cell grid.
        \item The runtime distribution can be seen here\ \ref{fig:runtime}.
    \end{itemize}

\begin{figure}[H]
    \centering
    \includegraphics[width=1.3\textwidth]{../../res/runtimeVisualized}
    \caption{Visualization of runtime distributions.}
    \label{fig:runtime}
\end{figure}

\section{Nano-scale flow simulation}
\label{sec:nano}

\subsection{Videos}
\label{subsec:vidnano}

    \begin{itemize}
        \item Here a list of all videos for the simulation.
        \item The simulation parameters were set to the specified values in the assignment sheet, if not stated otherwise. Further information can be found in the video descriptions
        \begin{itemize}
            \item Standard simulation run: \href{https://youtu.be/-eWISjhgIgA}{here}
            \item Addition of small cluster of particles: \href{https://youtu.be/G34H3SCnpW0}{here}
            \item Lennard-Jones parameters of the walls set to $\sigma = 0.8$ and $\epsilon = 5.0$: \href{https://youtu.be/I4h6tjnJVuI}{here}
            \item Thermostat frequency set to 1000, gravity constant set to -9.81, and the Lennard-Jones parameters of the walls set to $\sigma = 0.8$ and $\epsilon = 5.0$: \href{https://youtu.be/yxNYmXJg5r0}{here}
        \end{itemize}
    \end{itemize}

\subsection{Implementation of the flow simulation}
\label{subsec:wall}

    \begin{itemize}
        \item We implemented the fixation of particles to enable particle flow between walls.
        \item For that we had to introduce a new boolean attribute to particles to mark them as stationary.
        \item During cuboid generation, particles are added to a map representing static particles based on their type.
        \item A new thermostat class, \texttt{thermostat/IndividualThermostat}, was added to ignore fixated particles.
        \item The selection of the thermostat is now decided via a factory function. The thermostat can be turned off by selecting the \texttt{NoneThermostat}.
        \item This approach allowed us to run the simulation without creating a new simulation class and by simply specifying simulation parameters.
        \item Adjustments were made to the generation, boundary, and calculation files to accommodate stationary particles.
    \end{itemize}

\subsection{Analytics}
\label{subsec:ana}

    \begin{itemize}
        \item We added a class \texttt{analytics/Analyzer} that can write density and velocity profiles to a .csv file for later analysis.
        \item These profiles can then be plotted to a 2d heat map via a python script in \texttt{scripts/plots}.
        \item In the videos of\ \ref{subsec:vidnano} the plotted diagrams for density and velocity can be seen next to the running simulation.
    \end{itemize}

\end{document}
