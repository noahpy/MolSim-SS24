\section{Particle Generator}

\begin{frame}
    \frametitle{Abstracting Particle Clusters for simple Particle Generator}
    
    Particle clusters describe a collection of particles in a predefined structure

    \begin{itemize}
        \item \emph{ParticleCluster} is an abstract class requiring child classes to implement three methods:
        \begin{enumerate}
            \item (i) \emph{getTotalNumberOfParticles} $\rightarrow$ Returns the number of particles in the cluster
            \item (ii) \emph{generateCluster} $\rightarrow$ insert all particles into the passed particles vector
            \item (iii) \emph{toString} $\rightarrow$ Stringify for logging
        \end{enumerate}  
        \item We implemented a child class \emph{CuboidParticleCluster} which will generate a cluster as described on the problem sheet
        \item[] \!\!\!\!\!\!\!\!\! \textcolor{orange}{$\Rightarrow$} Abstraction allows for easy extension $\rightarrow$ "falling drop" sounds as if spherical clusters will become important
    \end{itemize}

    \begin{itemize}
        \item The \emph{ParticleGenerator} class has two methods to generate clusters of particles
        \begin{enumerate}
            \item \emph{registerCluster} $\rightarrow$ Add unique pointer to \emph{ParticleCluster} onto a vector 
            \item \emph{generateClusters} $\rightarrow$ Call \emph{generateCluster} on every registered \emph{ParticleCluster}
        \end{enumerate}
        \item[] \!\!\!\!\!\!\!\!\! \textcolor{orange}{$\Rightarrow$} Registering clusters first allows to know the total number of particles $\rightarrow$ allocate \textbf{one} vector 
    \end{itemize}

\end{frame}