
\section{Optimization}
\label{sec:opt}

\begin{frame}
    \frametitle{Base}
    \begin{itemize}
        \item Full Rayleigh-Taylor instability simulation run locally
    \end{itemize}
    \begin{table}[h!]
    \centering
    \begin{tabular}{|l|r|}
        \hline
        \textbf{Description} & \textbf{Value} \\ \hline
        Simulation ran for & 519 seconds (50,000 iterations) \\ \hline
        Average time per iteration & 10 ms \\ \hline
        MUP/S & 963,410 \\ \hline
    \end{tabular}
    \caption{Runtime for Rayleigh-Taylor simulation (MUP = force+vel+pos calc i.e. one update per particle per iteration). }
    \label{table:rayleigh_base}
    \end{table}
\end{frame}

\begin{frame}
    \frametitle{Inlining}
    \begin{itemize}
        \item Optimizations through inlining: Costs of function calls reduced
    \end{itemize}
    \begin{table}[h!]
    \centering
    \begin{tabular}{|l|r|}
        \hline
        \textbf{Description} & \textbf{Value} \\ \hline
        Simulation ran for & 454 seconds (50,000 iterations) \\ \hline
        Average time per iteration & 9 ms \\ \hline
        MUP/S & 1,101,343 \\ \hline
    \end{tabular}
    \caption{Runtime for raileigh-taylor simulation with inlining optimizations (MUP = force+vel+pos calc i.e. one update per particle per iteration).}
    \label{table:raileigh_inline}
    \end{table}
\end{frame}

\begin{frame}
    \frametitle{Calc optimizations}
    \begin{itemize}
        \item Reducing root operations with the usage of squared cutoff radii
        \item Applying gravity forces on container level instead of cell iteration level.
    \end{itemize}
    \begin{table}[h!]
        \centering
        \begin{tabular}{|l|r|}
            \hline
            \textbf{Description} & \textbf{Value} \\ \hline
            Simulation ran for & 440 seconds (50,000 iterations) \\ \hline
            Average time per iteration & 8 ms \\ \hline
            MUP/S & 1,136,386 \\ \hline
        \end{tabular}
        \caption{Runtime for raileigh-taylor simulation with force calculation optimizations (MUP = force+vel+pos calc i.e. one update per particle per iteration).}
        \label{table:raileigh_calc}
    \end{table}
\end{frame}